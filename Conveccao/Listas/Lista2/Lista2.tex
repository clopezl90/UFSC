\documentclass[]{article}
\usepackage[brazil]{babel}
\usepackage{graphicx}
\usepackage{mathtools}
\usepackage{float} 
\usepackage{xcolor}
\usepackage{amsmath, amssymb, bm}


\usepackage{array}
\usepackage{booktabs}



% margenes
\usepackage[a4paper,left=3cm,right=3cm,top=3cm]{geometry}

%opening
\title{}
\author{}

\begin{document}

\begin{center}
	{\tiny {\normalsize {\large Lista de exercicios 1\\
Convecçao\\
\textbf{Cristian Herledy Lopez Lara}}}}
\end{center}

\section*{Exercício 1}
\subsubsection*{1.1}

\textbf{Derivar, a partir de um volume infinitesimal, as equações da conservação da massa, momento e energia na forma diferencial.} \\


\textbf{Conservação da massa:} 

Considerando um volume de controle fixo dentro do campo de fluxo (com $\Delta x \Delta y  \Delta z \rightarrow 0 $), infinitesimalmente pequeno, a abordagem para conservação de massa será:

\begin{equation}
	\frac{\Delta m}{\Delta t} = \dot{m}_{entra} - \dot{m}_{sai}
\end{equation}
OBSERVAÇÃO: Para facilidade, os diagramas de equilíbrio são desenhados em duas dimensões.




Sua forma diferencial seria dada por

\begin{equation}
	\begin{aligned}
		\frac{\partial}{\partial t} (\rho \Delta x \Delta y \Delta z) = \rho u\Delta y\Delta z + \rho v\Delta x\Delta z + \rho w\Delta y\Delta z - [\rho u +	\frac{\partial}{\partial x}(\rho u)\Delta x]\Delta y\Delta z - \\ [\rho v +	\frac{\partial}{\partial y}(\rho v)\Delta y]\Delta x\Delta z - [\rho w +	\frac{\partial}{\partial z}(\rho w)\Delta z]\Delta x\Delta y
	\end{aligned}
\end{equation}

E dividindo pelo volume constante $\Delta x \Delta y  \Delta z$

\begin{equation}
	\begin{aligned}
		\frac{\partial}{\partial t} \rho + \frac{\partial}{\partial t} (\rho u) + \frac{\partial}{\partial t} (\rho v) + \frac{\partial}{\partial t} (\rho z) = 0 
	\end{aligned}
\end{equation}

Usando a definição de derivada material e o operador divergencia
\begin{equation}
	\begin{aligned}
		\frac{D}{D t} \rho +  \bigtriangledown (\rho V)= 0 
	\end{aligned}
\end{equation}


\textbf{Conservação do momento:} 

Derivando a análise da segunda lei de Newton e tomando a velocidade como uma propriedade de transporte, o abordagem para conservação do momento dentro do volume de controle (vc) será

\begin{equation}
	\frac{\Delta }{\Delta t} (mV)_{vc}= \sum F +\dot{m}V_{entra} - \dot{m}V_{sai}
\end{equation}




O equilíbrio de forças devido ao fluxo de momento e à tensão normal e tangencial na direção x será

\begin{equation}
	\begin{aligned}
		-\frac{\partial}{\partial t} (\rho u \Delta x \Delta y \Delta z) + \rho uu \Delta y \Delta z - [\rho uu +	\frac{\partial}{\partial x}(\rho uu)\Delta x]\Delta y\Delta z - \\  \rho uv \Delta x \Delta z - [\rho uv +	\frac{\partial}{\partial y}(\rho uv)\Delta y]\Delta x\Delta z -   \rho uw \Delta x \Delta y - [\rho uw +	\frac{\partial}{\partial z}(\rho uw)\Delta z]\Delta x\Delta y \\ + \sigma_{x} \Delta x\Delta y - (\sigma_{x} + \frac{\partial}{\partial x}\sigma_{x} \Delta x) \Delta y\Delta z - \tau_{xy} \Delta x\Delta y - (\tau_{xy} + \frac{\partial}{\partial x}\tau_{xy} \Delta y) \Delta x\Delta z + gx \Delta x\Delta y\Delta z
	\end{aligned}
\end{equation}

Sendo $gx$ o termo fonte. Dividindo pelo volume de controle quando $\Delta x \Delta y  \Delta z \rightarrow 0 $

\begin{equation}
	\begin{aligned}
		\rho \frac{ Du}{Dt} + u[ \frac{D}{D t} \rho + \rho(\frac{\partial}{\partial x}u \frac{\partial}{\partial y}v)] = - \frac{\partial \sigma_{x}}{\partial x} +  \frac{\partial \tau_{xy}}{\partial y} + gx   
	\end{aligned}
\end{equation}

Lembrando que o segundo termo é igual a zero segundo a conservação da massa

\begin{equation}
	\begin{aligned}
		\rho \frac{ Du}{Dt} = - \frac{\partial \sigma_{x}}{\partial x} +  \frac{\partial \tau_{xy}}{\partial y} + gx   
	\end{aligned}
\end{equation}

Com a definição do tensor de tensão dada por
\begin{equation}
	\tau_{ij} = -p \delta_{ij} + \mu \left( \frac{\partial u_i}{\partial x_j} + \frac{\partial u_j}{\partial x_i} \right) + \delta_{ij} \lambda \nabla \cdot \mathbf{V}
\end{equation}
\begin{equation}
	\lambda = -\frac{2}{3} \mu
\end{equation}

Que combinado com a equação 8 dá origem à equação de Navier-Stokes em x
\begin{equation}
	\begin{aligned}
	\frac{\partial}{\partial t} (\rho u) + \frac{\partial}{\partial x} (\rho u u) + \frac{\partial}{\partial y} (\rho v u) + \frac{\partial}{\partial z} (\rho w u) = \\
	- \frac{\partial p}{\partial x} + \frac{\partial}{\partial x} \left( 2 \mu \frac{\partial u}{\partial x} + \lambda \nabla \cdot \mathbf{V} \right) +
	\frac{\partial}{\partial y} \left( \mu \left( \frac{\partial u}{\partial y} + \frac{\partial v}{\partial x} \right) \right) +
	\frac{\partial}{\partial z} \left( \mu \left( \frac{\partial u}{\partial z} + \frac{\partial w}{\partial x} \right) \right) + g_x
	\end{aligned}
\end{equation}

E para $y$ e $z$

\begin{equation}
	\begin{aligned}
	\frac{\partial}{\partial t} (\rho v) + \frac{\partial}{\partial x} (\rho u v) + \frac{\partial}{\partial y} (\rho v v) + \frac{\partial}{\partial z} (\rho w v) = \\
	- \frac{\partial p}{\partial y} + \frac{\partial}{\partial x} \left( \mu \left( \frac{\partial v}{\partial x} + \frac{\partial u}{\partial y} \right) \right) +
	\frac{\partial}{\partial y} \left( 2 \mu \frac{\partial v}{\partial y} + \lambda \nabla \cdot \mathbf{V} \right) +
	\frac{\partial}{\partial z} \left( \mu \left( \frac{\partial v}{\partial z} + \frac{\partial w}{\partial y} \right) \right) +  g_y
	\end{aligned}
\end{equation}

\begin{equation}
	\begin{aligned}
	\frac{\partial}{\partial t} (\rho w) + \frac{\partial}{\partial x} (\rho u w) + \frac{\partial}{\partial y} (\rho v w) + \frac{\partial}{\partial z} (\rho w w) = \\
	- \frac{\partial p}{\partial z} + \frac{\partial}{\partial x} \left( \mu \left( \frac{\partial w}{\partial x} + \frac{\partial u}{\partial z} \right) \right) +
	\frac{\partial}{\partial y} \left( \mu \left( \frac{\partial w}{\partial y} + \frac{\partial v}{\partial z} \right) \right) +
	\frac{\partial}{\partial z} \left( 2 \mu \frac{\partial w}{\partial z} + \lambda \nabla \cdot \mathbf{V} \right) +  g_z
	\end{aligned}
\end{equation}

\textbf{Conservação da energia:} 

Da mesma forma, o diagrama a seguir mostra a análise da primeira lei da termodinâmica para volume infinitesimal





\begin{equation}
	\begin{aligned}
		\rho \frac{ De}{Dt} + e(\frac{D}{D t} \rho +  \bigtriangledown (\rho V)) = -\bigtriangledown q'' + q''' - P\bigtriangledown V 
	\end{aligned}
\end{equation}


\begin{equation}
	\begin{aligned}
		X = cb^{\frac{3}{4}}tanh(cb^{\frac{3}{4}}) 
	\end{aligned}
\end{equation}


\end{document}





